\documentclass[a4paper,url]{article}

\usepackage[utf8]{inputenc}
\usepackage[T1]{fontenc}
\usepackage[german]{babel}
\usepackage[small]{subfigure}
\usepackage{amsmath}
\usepackage{enumerate}
\usepackage{graphicx}
\usepackage{ifthen}
\usepackage{amssymb}
\usepackage{hyperref}
\usepackage{tikz}

% Change dimension of a page
\usepackage{geometry}
\geometry{a4paper, left = 30mm, right=30mm, top=30mm, bottom=30mm, headheight=1mm, headsep=1mm, footskip=15mm}


%Algorithms
\usepackage[ruled,linesnumbered]{algorithm2e}

\newcommand {\rpf}{\begin{math}\rightarrow\end{math}}
\newcommand {\ra}{\rightarrow}
\newcommand {\epsi}{\begin{math}\epsilon\end{math}}

\newcommand{\bbN}{\mathbb{N}}
\newcommand{\bbZ}{\mathbb{Z}}
\newcommand{\bbQ}{\mathbb{Q}}
\newcommand{\bbR}{\mathbb{R}}
\newcommand{\calO}{\mathcal{O}}
\newcommand{\bfE}{\mathbf{E}}
\newcommand{\bfP}{\mathbf{P}}
\DeclareMathOperator{\var}{\bf{var}}

\selectlanguage{german}

\newcounter{aufgabe_count}
\setcounter{aufgabe_count}{1}
\newcommand{\aufgabe}[2]{\vspace{3.5ex} {\noindent \bf\large Aufgabe
\arabic{aufgabe_count}: \hspace{10pt}#1} \hspace{5pt}(#2 Punkte)\vspace{3pt}\\ 
\stepcounter{aufgabe_count} 
}

\begin{document}
\textheight55\baselineskip

\pagestyle{plain}
\pagenumbering{arabic}

\noindent
\begin{minipage}[t]{0.6\textwidth}
\begin{flushleft}
\bf Übungen zu Algorithmen\\
WSI für Informatik\\
Kaufmann/Bekos/Schneck
\end{flushleft}
\end{minipage}
\begin{minipage}[t]{0.4\textwidth}
\begin{flushright}
\bf Sommersemester 2018\\
Universität Tübingen\\
18.05.2018 %@@@Datum eintragen
\end{flushright}
\end{minipage}

\vspace{5.0ex}
\noindent

\centerline{\huge \bf Übungsblatt 5}
\vspace{1ex}
\centerline{\bf Abgabe bis 31.05.2018}
\vspace{.5ex}
\centerline{\bf Besprechung: 04.06.2018 -- 07.06.2018}

\aufgabe{Bucketsort}{3}
Gegeben seien $n$ Zahlen im Bereich $[0,\dots, n^k-1]$ für ein festes $k\in \mathbb N$.
Zeigen Sie, dass mittels Bucketsort die Zahlen mit einer Worstcase-Laufzeit von $O(n)$ sortiert werden können.\\
\textit{Hinweis: Stellen Sie die Zahlen zur Basis $n$ dar.}



\aufgabe{Mediansuche in Linearzeit}{3 + 6}
\begin{enumerate}[(a)]
\item In der Vorlesung haben Sie gelernt, wie Sie den Median von fünf
  Zahlen mit 7 Vergleichen bestimmen können. Verallgemeinern Sie
  dieses (rekursive) Vorgehen für Eingaben ungerader Länge. Geben Sie
  die Anzahl benötigter Vergleiche für Listen der Länge $n$, mit
  $n\in \{7,9,11,13,15,17\}$ an.
\item Für die Laufzeitanalyse der Medianbestimmung ergab sich bei einer Unterteilung in Fünfergruppen die folgende Rekursionsgleichung:
\[
T(n) \leq T
\left(
  \frac{n}{5}
\right) + T
\left(
  \frac{7}{10}n
\right) + c \cdot n
\]
\begin{enumerate}[i.]
\item Geben Sie eine möglichst gute Konstante $c$ an.
\item Wie verändert sich die Rekursionsgleichung, wenn die Liste in Siebener- statt in Fünfergruppen geteilt wird?
Geben Sie auch hier eine möglichst gute Abschätzung der Konstante $c$ an.
\end{enumerate}
\end{enumerate}




\aufgabe{Median in sortierten Arrays}{3 + 3}
Seien $A[1..n]$ und $B[1..n]$ zwei sorrtierte Arrays der Größe $n$.
\begin{enumerate}[(a)]
\item Beschreiben Sie einen Algorithmus, der den Median aller $2n$ Elemente aus $A$ und $B$ in Zeit $\calO(\log n)$ findet.
\item Begründen Sie die Laufzeit und die Korrektheit Ihres Algorithmus.
\end{enumerate}
Für die Analyse Ihres Algorithmus dürfen Sie annehmen, dass $n$ eine Zweierpotenz ist.

\newpage

\aufgabe{Konvexe Hülle}{4}
Gegeben sei eine Menge $M=\{(x_i, y_i)\in \mathbb Q^2 \mid 1 \leq i \leq n\}$. Die konvexe Hülle $\operatorname{conv}(M)$ einer Menge $M$ ist der kleinste Polyeder der alle Punkte in $M$ enthält. Eine einfache Repräsentation der konvexen Hülle ist die kleinste Menge $N \subseteq M$ von Punkten, so dass $\operatorname{conv}(N)=\operatorname{conv}(M)$. Bildlich gesprochen: die \textit{Eckpunkte} der Menge $M$. Eine \textit{zyklische Sortierung} der Eckpunkte ist eine Ordnung der Punkte in $N$ so, dass beim Ablaufen der Punkte nach der Sortierung ein Weg um das Polygon entsteht.

\medskip\noindent Nehmen Sie an, dass Sie einen Algorithmus haben der in Zeit $T(n)$, gegeben eine Menge $M$ (wie oben), eine zyklische Sortierung der Eckpunkte der konvexen Hülle ausgibt. Zeigen Sie, dass $\Omega(n \log n)$ eine untere Schranke für die worst-case Laufzeit des Algorithmus ist.

\begin{figure}[htp!]
\begin{center}
  \begin{tikzpicture}
    \draw[fill] (0,0) circle(1.5pt) node[left]{$x_1$};
    \draw[fill] (1,1) circle(1.5pt) node[left]{$x_2$};
    \draw[fill] (.5,2) circle(1.5pt) node[left]{$x_6$};
    \draw[fill] (2,1) circle(1.5pt) node[right]{$x_3$};
    \draw[fill] (-1,3) circle(1.5pt) node[left]{$x_4$};
    \draw[fill] (2,3) circle(1.5pt) node[right]{$x_5$};   
    \draw (0,0) -- (2,1) -- (2,3) -- (-1,3)-- (0,0);
    \draw[fill=black, opacity=.2] (0,0) -- (2,1) -- (2,3) -- (-1,3)-- (0,0);
  \end{tikzpicture}\\
\end{center}
\caption{Der Algorithmus erhält die Punktemenge $\{x_1,\dots,x_6\}$ und gibt die Punkteliste $(x_1,x_4,x_5,x_3)$ aus. Der grau gefärbte Bereich ist die konvexe Hülle.
}\end{figure}


\aufgabe{Untere Schranken}{3 + 5}
Sei $A$ eine sortierte Liste der Größe $n$.
\begin{enumerate}[(a)]
\item Das erste Problem besteht daraus, zu entscheiden, ob in der Liste $A$ Duplikate vorkommen.
Beweisen Sie, dass $n-1$ eine untere Schranke für die Anzahl der Vergleiche ist, die man dafür braucht.
\item Das zweite Problem ist es, zu entscheiden, ob in der Liste $A$ ein beliebiger Wert $x$ vorkommt.
Beweisen Sie, dass im worst-case $\Omega(\log n)$ Vergleiche dafür benötigt werden (unabhängig vom verwendeten
Algorithmus).
\end{enumerate}


\end{document}
